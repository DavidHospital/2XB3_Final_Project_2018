\documentclass[12pt]{article}

\usepackage{graphicx}
\usepackage{paralist}
\usepackage{amsfonts}
\usepackage{amsmath}
\usepackage{hhline}
\usepackage{booktabs}
\usepackage{multirow}
\usepackage{multicol}

\oddsidemargin 0mm
\evensidemargin 0mm
\textwidth 160mm
\textheight 200mm
\renewcommand\baselinestretch{1.0}

\pagestyle {plain}
\pagenumbering{arabic}

\newcounter{stepnum}

%% Comments

\usepackage{color}

\begin {document}

\section* {Disaster Event Module}

\subsection* {Module}

DisasterEvent

\subsection* {Uses}

LatLng (android)

\subsection* {Syntax}

\subsubsection* {Exported Types}

DisasterEvent = ?

\subsubsection* {Exported Access Programs}

\begin{tabular}{| l | l | l | l |}
\hline
\textbf{Routine name} & \textbf{In} & \textbf{Out} & \textbf{Exceptions}\\
\hline
DisasterEvent & String, LatLng, LatLng, $\mathbb{Z}$, $\mathbb{Z}$, $\mathbb{Z}$ & DisasterEvent & \\
\hline
compareTo & DisasterEvent & $\mathbb{Z}$ & ~ \\
\hline
getType & ~ & String & ~ \\
\hline
getLocation1 & ~ & LatLng & ~ \\
\hline
getLocation2 & ~ & LatLng & ~ \\
\hline
getYear & ~ & $\mathbb{Z}$ & ~ \\
\hline
getMonth & ~ & $\mathbb{Z}$ & ~ \\
\hline
getDay & ~ & $\mathbb{Z}$ & ~ \\
\hline
\end{tabular}

\subsection* {Semantics}

\subsubsection* {State Variables}

type: String \\
location1: LatLng \\
location2: LatLng \\
year: $\mathbb{Z}$ \\
month: $\mathbb{Z}$ \\
day: $\mathbb{Z}$ \\

\subsubsection* {State Invariant}

None

\subsubsection* {Assumptions}

The constructor DisasterEvent is called for each object instance before any other access routine is called for that object. The constructor cannot be called on an existing object.

\subsubsection* {Access Routine Semantics}

DisasterEvent($t, l_1, l_2, y, m, d$):
\begin{itemize}
\item transition: $type, location1, location2, year, month, day := t, l_1, l_2, y, m, d$
\item output: $out := \mathit{self}$
\end{itemize}

\noindent compareTo($other$):
\begin{itemize}
\item output: $out := year < other.year \Rightarrow -1 | year > other.year \Rightarrow 1 | (month < other.month \Rightarrow -1 | month > other.month \Rightarrow 1 | (day < other.day \Rightarrow -1 | day > other.day \Rightarrow 1 | 0))$
\end{itemize}

\noindent getType():
\begin{itemize}
\item output: $out := type$
\end{itemize}

\noindent getLocation1():
\begin{itemize}
\item output: $out := location1$
\end{itemize}

\noindent getLocation2():
\begin{itemize}
\item output: $out := location2$
\end{itemize}

\noindent getYear():
\begin{itemize}
\item output: $out := year$
\end{itemize}

\noindent getMonth():
\begin{itemize}
\item output: $out := month$
\end{itemize}

\noindent getDay():
\begin{itemize}
\item output: $out := day$
\end{itemize}

\newpage





\section* {Data Module}

\subsection* {Module}

Data

\subsection* {Uses}

DisasterEvent

\subsection* {Syntax}

\subsubsection* {Exported Types}

None

\subsubsection* {Exported Access Programs}

\begin{tabular}{| l | l | l | l |}
\hline
\textbf{Routine name} & \textbf{In} & \textbf{Out} & \textbf{Exceptions}\\
\hline
clear & ~ & ~ & ~ \\
\hline
add & DisasterEvent & ~ & ~ \\
\hline
getList & String & Sequence of DisasterEvent & ~ \\
\hline
getTypes & ~ & Set of DisasterEvent & ~ \\
\hline
\end{tabular}

\subsection* {Semantics}

\subsubsection* {State Variables}

data: HashMap of (String, Sequence of DisasterEvent)

\subsubsection* {State Invariant}

None

\newpage
\subsubsection* {Access Routine Semantics}

clear():
\begin{itemize}
\item transition: $data := <>$
\end{itemize}

\noindent add(de):
\begin{itemize}
\item transition: $data.\mbox{getList}(de.\mbox{getType}) = data.\mbox{getList}(de.\mbox{getType}) \| <de>$
\end{itemize}

\noindent getList(key):
\begin{itemize}
\item output: $out := data.\mbox{get}(key)$
\end{itemize}

\noindent getTypes():
\begin{itemize}
\item output: $out := data.\mbox{keySet}$
\end{itemize}

\newpage





\section* {New Parser Module}

\subsection* {Module}

NewParser

\subsection* {Uses}

Data, DisasterEvent

\subsection* {Syntax}

\subsubsection* {Exported Types}

None

\subsubsection* {Exported Access Programs}

\begin{tabular}{| l | l | l | l |}
\hline
\textbf{Routine name} & \textbf{In} & \textbf{Out} & \textbf{Exceptions}\\
\hline
firstParser & String & ~ & ~ \\
\hline
secondParser & String & ~ & ~ \\
\hline
\end{tabular}

\subsection* {Semantics}

\subsubsection* {State Variables}

None

\subsubsection* {State Invariant}

None

\newpage
\subsubsection* {Access Routine Semantics}

firstParser($s$):
\begin{itemize}
\item Opens a file $f$ with name $s$. For each row, let $r$ be an array of strings, representing the columns defined in $f$. Let $year, month, day, type, lat1, lng1, lat2, lng2$ := $r[0]$.substring(0, 4), $r[0]$.substring(4, 6), $r[1]$, $r[12]$, $r[44]$, $r[45]$,$r[46]$, $r[47]$. Open a second file $f'$ with name ``c'' $\|s$. For each row in $f$, write to $f'$ the line:\\
$year\|month\|day\|type\|lat1\|lng1\|lat2\|lng2$
\end{itemize}

secondParser($s$):
\begin{itemize}
\item Used to parse files created from firstParser
\item Opens a file $f$ with name $s$. For each row, let $r$ be an array of strings, representing the columns defined in $f$. Let $year, month, day, type, lat1, lng1, lat2, lng2$ := $r[0]$, $r[1]$, $r[2]$, $r[3]$, $r[4]$, $r[5]$,$r[6]$, $r[7]$. For each row in $f$, \\let $de := \mbox{DisasterEvent}(year, month, day, type, lat1, lng1, lat2, lng2)$. 
$\mbox{Data}.\mbox{add}(de)$
\end{itemize}




\end {document}